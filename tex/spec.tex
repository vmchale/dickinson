\documentclass{report}

\usepackage{amsmath}
\usepackage{syntax}

\begin{document}

\title{Dickinson Language Reference}
\author {Vanessa McHale}
\maketitle

\tableofcontents

\section{Introduction}

Dickinson is a language for generative literature targeting English. This reference specifies the syntax of the language.

\section{Syntax}

\subsection{Lexical Structure}

Dickinson programs have the following lexical structure:

\begin{grammar}

    <comment> ::= ";."*"\$"

    <identifier> ::= ["a"-"z"]["A"-"Za"-"z0"-"9"]*

    <typeIdentifier> ::= ["A"-"Z"]["A"-"Za"-"z0"-"9"]*

    <moduleIdentifier> ::= (<identifier>".")* <identifier>

    <probability> ::= (["0"-"9"]+|["0"-"9"]+"."["0"-"9"]*)

\end{grammar}

\subsection{Syntax Tree}

\setlength{\grammarparsep}{20pt plus 1pt minus 1pt}
\setlength{\grammarindent}{12em}

% how does dhall do it?

\begin{grammar}
<pattern> ::= "_"
\alt <identifier>
\alt <typeIdentifier>
\alt <pattern> ("|" <pattern>)+
\alt (<pattern> ("," <pattern>)+)

<type> ::= "text"
\alt ($\rightarrow$ <type> <type>)
\alt (<type> (, <type>)*)
\alt <identifier>

<expression> ::= <string>
\alt ("let:" [(<identifier> <expression>)+] <expression>)
\alt ("bind:" [(<identifier> <expression>)+] <expression>)
\alt (<expression> (, <expression>)*)
\alt (":flatten" <expression>)
\alt (<expression> : <type>)
\alt <typeIdentifier>
\alt (":pick" <identifier>)
\alt (">" <expression>*)
\alt (":oneof" ("|" <expression>)+)
\alt (":branch" ("|" <probability> <expression>)+)
\alt ("$" <expression> <expression>)
\alt (":match" <expression> [(<pattern> <expression>)+])

<declaration> ::= (":def" <identifier> <expression>)
\alt "tydecl" <identifier> "=" <typeIdentifier> ("|" <typeIdentifier>)+

<include> ::= (":include" <moduleIdentifier>)

<module> ::= <include>* "%-" <declaration>*
\end{grammar}

\end{document}
